\chapter*{Introduction}\label{cha:Introduction}
The lung model introduced here serves the purpose of simulating multiple-breath washout (MBW) and single breath washout (SBW), two lung function test commonly performed in pediatric pneumology \citep{Robinson2013, Singer2013}.
Inert gas washouts with one tracer gas (e.g. $N_2$-MBW) or two tracer gases ($SF_6$-$He$ double tracer gas SBW) can be simulated.

At the core of the lung model are two solvers: A lumped parameter model (also referred to as 0D model) for the ventilation within the lung, and a 1D network model for gas transport.
The morphology of the model is defined by a tree-like network of bifurcating airways (mainly for the conducting airways), ending in trumpet-like compliant airway units.
In mathematical terms, the ventilation solves a linear system for the pressure $p$ in each bifurcation node and in the pleural gap, and in subsequent step solves for the flow rates $Q$ between two nodes, at discrete time steps.
The 1D model solves an advection-diffusion transport equation for one (or optional two) scalar (normalized) quantity $c$ on 1D grid along all airways units.
The two solvers are coupled through the flow-rate $Q$ resulting from the ventilation lumped parameter model and which is used to derive the advection velocity $u$ used in the transport equation.
A set of model parameters, defining the material and structural properties of the lung, are at hand, most of which can be modified through dedicated tables and input files.

The current model serves well to put into relation, structural and mechanical properties and variations/modification of these, with features of the gas washout profile.
In that sense, effects of certain types of disease and conditions on the lung function can be simulated and analysed in a qualitative manner.
However, with the current state of the model, it is not possible to simulate all types of airway disease and analyse the their outcomes in an absolute and quantitative manner.
